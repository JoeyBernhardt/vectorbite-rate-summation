\documentclass[]{article}
\usepackage{lmodern}
\usepackage{amssymb,amsmath}
\usepackage{ifxetex,ifluatex}
\usepackage{fixltx2e} % provides \textsubscript
\ifnum 0\ifxetex 1\fi\ifluatex 1\fi=0 % if pdftex
  \usepackage[T1]{fontenc}
  \usepackage[utf8]{inputenc}
\else % if luatex or xelatex
  \ifxetex
    \usepackage{mathspec}
  \else
    \usepackage{fontspec}
  \fi
  \defaultfontfeatures{Ligatures=TeX,Scale=MatchLowercase}
\fi
% use upquote if available, for straight quotes in verbatim environments
\IfFileExists{upquote.sty}{\usepackage{upquote}}{}
% use microtype if available
\IfFileExists{microtype.sty}{%
\usepackage{microtype}
\UseMicrotypeSet[protrusion]{basicmath} % disable protrusion for tt fonts
}{}
\usepackage[margin=1in]{geometry}
\usepackage{hyperref}
\hypersetup{unicode=true,
            pdftitle={Covariance messing around},
            pdfborder={0 0 0},
            breaklinks=true}
\urlstyle{same}  % don't use monospace font for urls
\usepackage{color}
\usepackage{fancyvrb}
\newcommand{\VerbBar}{|}
\newcommand{\VERB}{\Verb[commandchars=\\\{\}]}
\DefineVerbatimEnvironment{Highlighting}{Verbatim}{commandchars=\\\{\}}
% Add ',fontsize=\small' for more characters per line
\usepackage{framed}
\definecolor{shadecolor}{RGB}{248,248,248}
\newenvironment{Shaded}{\begin{snugshade}}{\end{snugshade}}
\newcommand{\KeywordTok}[1]{\textcolor[rgb]{0.13,0.29,0.53}{\textbf{{#1}}}}
\newcommand{\DataTypeTok}[1]{\textcolor[rgb]{0.13,0.29,0.53}{{#1}}}
\newcommand{\DecValTok}[1]{\textcolor[rgb]{0.00,0.00,0.81}{{#1}}}
\newcommand{\BaseNTok}[1]{\textcolor[rgb]{0.00,0.00,0.81}{{#1}}}
\newcommand{\FloatTok}[1]{\textcolor[rgb]{0.00,0.00,0.81}{{#1}}}
\newcommand{\ConstantTok}[1]{\textcolor[rgb]{0.00,0.00,0.00}{{#1}}}
\newcommand{\CharTok}[1]{\textcolor[rgb]{0.31,0.60,0.02}{{#1}}}
\newcommand{\SpecialCharTok}[1]{\textcolor[rgb]{0.00,0.00,0.00}{{#1}}}
\newcommand{\StringTok}[1]{\textcolor[rgb]{0.31,0.60,0.02}{{#1}}}
\newcommand{\VerbatimStringTok}[1]{\textcolor[rgb]{0.31,0.60,0.02}{{#1}}}
\newcommand{\SpecialStringTok}[1]{\textcolor[rgb]{0.31,0.60,0.02}{{#1}}}
\newcommand{\ImportTok}[1]{{#1}}
\newcommand{\CommentTok}[1]{\textcolor[rgb]{0.56,0.35,0.01}{\textit{{#1}}}}
\newcommand{\DocumentationTok}[1]{\textcolor[rgb]{0.56,0.35,0.01}{\textbf{\textit{{#1}}}}}
\newcommand{\AnnotationTok}[1]{\textcolor[rgb]{0.56,0.35,0.01}{\textbf{\textit{{#1}}}}}
\newcommand{\CommentVarTok}[1]{\textcolor[rgb]{0.56,0.35,0.01}{\textbf{\textit{{#1}}}}}
\newcommand{\OtherTok}[1]{\textcolor[rgb]{0.56,0.35,0.01}{{#1}}}
\newcommand{\FunctionTok}[1]{\textcolor[rgb]{0.00,0.00,0.00}{{#1}}}
\newcommand{\VariableTok}[1]{\textcolor[rgb]{0.00,0.00,0.00}{{#1}}}
\newcommand{\ControlFlowTok}[1]{\textcolor[rgb]{0.13,0.29,0.53}{\textbf{{#1}}}}
\newcommand{\OperatorTok}[1]{\textcolor[rgb]{0.81,0.36,0.00}{\textbf{{#1}}}}
\newcommand{\BuiltInTok}[1]{{#1}}
\newcommand{\ExtensionTok}[1]{{#1}}
\newcommand{\PreprocessorTok}[1]{\textcolor[rgb]{0.56,0.35,0.01}{\textit{{#1}}}}
\newcommand{\AttributeTok}[1]{\textcolor[rgb]{0.77,0.63,0.00}{{#1}}}
\newcommand{\RegionMarkerTok}[1]{{#1}}
\newcommand{\InformationTok}[1]{\textcolor[rgb]{0.56,0.35,0.01}{\textbf{\textit{{#1}}}}}
\newcommand{\WarningTok}[1]{\textcolor[rgb]{0.56,0.35,0.01}{\textbf{\textit{{#1}}}}}
\newcommand{\AlertTok}[1]{\textcolor[rgb]{0.94,0.16,0.16}{{#1}}}
\newcommand{\ErrorTok}[1]{\textcolor[rgb]{0.64,0.00,0.00}{\textbf{{#1}}}}
\newcommand{\NormalTok}[1]{{#1}}
\usepackage{graphicx,grffile}
\makeatletter
\def\maxwidth{\ifdim\Gin@nat@width>\linewidth\linewidth\else\Gin@nat@width\fi}
\def\maxheight{\ifdim\Gin@nat@height>\textheight\textheight\else\Gin@nat@height\fi}
\makeatother
% Scale images if necessary, so that they will not overflow the page
% margins by default, and it is still possible to overwrite the defaults
% using explicit options in \includegraphics[width, height, ...]{}
\setkeys{Gin}{width=\maxwidth,height=\maxheight,keepaspectratio}
\IfFileExists{parskip.sty}{%
\usepackage{parskip}
}{% else
\setlength{\parindent}{0pt}
\setlength{\parskip}{6pt plus 2pt minus 1pt}
}
\setlength{\emergencystretch}{3em}  % prevent overfull lines
\providecommand{\tightlist}{%
  \setlength{\itemsep}{0pt}\setlength{\parskip}{0pt}}
\setcounter{secnumdepth}{0}
% Redefines (sub)paragraphs to behave more like sections
\ifx\paragraph\undefined\else
\let\oldparagraph\paragraph
\renewcommand{\paragraph}[1]{\oldparagraph{#1}\mbox{}}
\fi
\ifx\subparagraph\undefined\else
\let\oldsubparagraph\subparagraph
\renewcommand{\subparagraph}[1]{\oldsubparagraph{#1}\mbox{}}
\fi

%%% Use protect on footnotes to avoid problems with footnotes in titles
\let\rmarkdownfootnote\footnote%
\def\footnote{\protect\rmarkdownfootnote}

%%% Change title format to be more compact
\usepackage{titling}

% Create subtitle command for use in maketitle
\providecommand{\subtitle}[1]{
  \posttitle{
    \begin{center}\large#1\end{center}
    }
}

\setlength{\droptitle}{-2em}

  \title{Covariance messing around}
    \pretitle{\vspace{\droptitle}\centering\huge}
  \posttitle{\par}
    \author{}
    \preauthor{}\postauthor{}
    \date{}
    \predate{}\postdate{}
  

\begin{document}
\maketitle

\begin{Shaded}
\begin{Highlighting}[]
\NormalTok{knitr::opts_chunk$}\KeywordTok{set}\NormalTok{(}\DataTypeTok{message =} \OtherTok{FALSE}\NormalTok{)}
\NormalTok{knitr::opts_chunk$}\KeywordTok{set}\NormalTok{(}\DataTypeTok{warning =} \OtherTok{FALSE}\NormalTok{)}
\end{Highlighting}
\end{Shaded}

\begin{Shaded}
\begin{Highlighting}[]
\KeywordTok{library}\NormalTok{(tidyverse)}
\KeywordTok{library}\NormalTok{(cowplot)}
\KeywordTok{theme_set}\NormalTok{(}\KeywordTok{theme_cowplot}\NormalTok{())}
\KeywordTok{library}\NormalTok{(roll)}
\end{Highlighting}
\end{Shaded}

\paragraph{Invent some TPCs}\label{invent-some-tpcs}

\begin{Shaded}
\begin{Highlighting}[]
\NormalTok{f_curve <-}\StringTok{ }\NormalTok{function(x) }\FloatTok{0.06}\NormalTok{*}\KeywordTok{exp}\NormalTok{(}\FloatTok{0.09}\NormalTok{*x)*(}\DecValTok{1}\NormalTok{-((x}\DecValTok{-15}\NormalTok{)/(}\DecValTok{34}\NormalTok{/}\DecValTok{2}\NormalTok{))^}\DecValTok{2}\NormalTok{)}
\NormalTok{g_curve <-}\StringTok{ }\NormalTok{function(x) }\FloatTok{0.01}\NormalTok{*}\KeywordTok{exp}\NormalTok{(}\FloatTok{0.15}\NormalTok{*x)}
\NormalTok{p <-}\StringTok{ }\KeywordTok{ggplot}\NormalTok{(}\DataTypeTok{data =} \KeywordTok{data.frame}\NormalTok{(}\DataTypeTok{x =} \DecValTok{0}\NormalTok{), }\DataTypeTok{mapping =} \KeywordTok{aes}\NormalTok{(}\DataTypeTok{x =} \NormalTok{x)) }

\NormalTok{tpcs <-}\StringTok{ }\NormalTok{p +}\StringTok{ }
\StringTok{    }\KeywordTok{stat_function}\NormalTok{(}\DataTypeTok{fun =} \NormalTok{f_curve, }\DataTypeTok{color =} \StringTok{"purple"}\NormalTok{, }\DataTypeTok{size =} \DecValTok{2}\NormalTok{) +}
\StringTok{    }\KeywordTok{stat_function}\NormalTok{(}\DataTypeTok{fun =} \NormalTok{g_curve, }\DataTypeTok{color =} \StringTok{"orange"}\NormalTok{, }\DataTypeTok{size =} \DecValTok{2}\NormalTok{) +}
\StringTok{    }\KeywordTok{xlim}\NormalTok{(}\DecValTok{0}\NormalTok{, }\DecValTok{35}\NormalTok{) +}\StringTok{ }\KeywordTok{ylim}\NormalTok{(-}\FloatTok{0.1}\NormalTok{, }\FloatTok{0.5}\NormalTok{) +}\StringTok{ }
\StringTok{    }\KeywordTok{xlab}\NormalTok{(}\StringTok{"Temperature (°C)"}\NormalTok{) +}\StringTok{ }\KeywordTok{ylab}\NormalTok{(}\StringTok{"Trait"}\NormalTok{) +}\StringTok{ }\KeywordTok{geom_hline}\NormalTok{(}\DataTypeTok{yintercept =} \DecValTok{0}\NormalTok{)}

\NormalTok{tpcs}
\end{Highlighting}
\end{Shaded}

\includegraphics{08_covariance_messing_around_files/figure-latex/unnamed-chunk-3-1.pdf}

\paragraph{make up temperature time
series}\label{make-up-temperature-time-series}

\begin{Shaded}
\begin{Highlighting}[]
\KeywordTok{set.seed}\NormalTok{(}\DecValTok{101}\NormalTok{)}
\NormalTok{time <-}\StringTok{ }\KeywordTok{seq_along}\NormalTok{(}\DecValTok{1}\NormalTok{:}\DecValTok{100}\NormalTok{)}
\NormalTok{temps <-}\StringTok{ }\KeywordTok{data.frame}\NormalTok{(}\DataTypeTok{temperature =} \KeywordTok{runif}\NormalTok{(}\DecValTok{100}\NormalTok{, }\DataTypeTok{min =} \DecValTok{5}\NormalTok{, }\DataTypeTok{max =} \DecValTok{33}\NormalTok{), }\DataTypeTok{time =} \NormalTok{time)}
\end{Highlighting}
\end{Shaded}

\paragraph{plot temps over time}\label{plot-temps-over-time}

\begin{Shaded}
\begin{Highlighting}[]
\NormalTok{temp_plot <-}\StringTok{ }\NormalTok{temps %>%}\StringTok{ }
\StringTok{    }\KeywordTok{ggplot}\NormalTok{(}\KeywordTok{aes}\NormalTok{(}\DataTypeTok{x =} \NormalTok{time, }\DataTypeTok{y =} \NormalTok{temperature)) +}\StringTok{ }\KeywordTok{geom_line}\NormalTok{(}\DataTypeTok{size =} \DecValTok{1}\NormalTok{) +}
\StringTok{    }\KeywordTok{ylab}\NormalTok{(}\StringTok{"Temperature (°C)"}\NormalTok{) +}\StringTok{ }\KeywordTok{xlab}\NormalTok{(}\StringTok{"Time (days)"}\NormalTok{)}

\NormalTok{temp_plot}
\end{Highlighting}
\end{Shaded}

\includegraphics{08_covariance_messing_around_files/figure-latex/unnamed-chunk-5-1.pdf}

\paragraph{get trait values}\label{get-trait-values}

\begin{Shaded}
\begin{Highlighting}[]
\NormalTok{f_time_series <-}\StringTok{ }\KeywordTok{data.frame}\NormalTok{(}\DataTypeTok{time =} \NormalTok{time, }\DataTypeTok{ftrait =} \KeywordTok{sapply}\NormalTok{(}\DataTypeTok{X =} \NormalTok{temps$temperature, }\DataTypeTok{FUN =} \NormalTok{f_curve), }\DataTypeTok{temperature =} \NormalTok{temps$temperature)}
\NormalTok{g_time_series <-}\StringTok{ }\KeywordTok{data.frame}\NormalTok{(}\DataTypeTok{time =} \NormalTok{time, }\DataTypeTok{gtrait =} \KeywordTok{sapply}\NormalTok{(}\DataTypeTok{X =} \NormalTok{temps$temperature, }\DataTypeTok{FUN =} \NormalTok{g_curve), }\DataTypeTok{temperature =} \NormalTok{temps$temperature)}
\end{Highlighting}
\end{Shaded}

\paragraph{plot traits over time}\label{plot-traits-over-time}

\begin{Shaded}
\begin{Highlighting}[]
\NormalTok{both <-}\StringTok{ }\KeywordTok{left_join}\NormalTok{(f_time_series, g_time_series) %>%}\StringTok{ }
\StringTok{    }\KeywordTok{mutate}\NormalTok{(}\DataTypeTok{R_0 =} \NormalTok{ftrait*gtrait)}


\NormalTok{trait_plot <-}\StringTok{ }\NormalTok{both %>%}\StringTok{ }
\StringTok{    }\KeywordTok{gather}\NormalTok{(}\DataTypeTok{key =} \NormalTok{trait_type, }\DataTypeTok{value =} \NormalTok{trait_value, ftrait, gtrait) %>%}\StringTok{ }
\StringTok{    }\KeywordTok{ggplot}\NormalTok{(}\KeywordTok{aes}\NormalTok{(}\DataTypeTok{x =} \NormalTok{time, }\DataTypeTok{y =} \NormalTok{trait_value, }\DataTypeTok{color =} \NormalTok{trait_type)) +}\StringTok{ }\KeywordTok{geom_line}\NormalTok{(}\DataTypeTok{size =} \DecValTok{1}\NormalTok{) +}
\StringTok{    }\KeywordTok{scale_color_manual}\NormalTok{(}\DataTypeTok{values =} \KeywordTok{c}\NormalTok{(}\StringTok{"purple"}\NormalTok{, }\StringTok{"orange"}\NormalTok{)) +}\KeywordTok{ylab}\NormalTok{(}\StringTok{"Trait value"}\NormalTok{) +}\StringTok{ }\KeywordTok{xlab}\NormalTok{(}\StringTok{"Time (days)"}\NormalTok{) +}
\StringTok{    }\KeywordTok{theme}\NormalTok{(}\DataTypeTok{legend.position =} \StringTok{"top"}\NormalTok{)}

\NormalTok{trait_plot}
\end{Highlighting}
\end{Shaded}

\includegraphics{08_covariance_messing_around_files/figure-latex/unnamed-chunk-7-1.pdf}

\paragraph{Calculate rolling covariances over the time series with
different time slice durations (2, 5, 10, 20
days)}\label{calculate-rolling-covariances-over-the-time-series-with-different-time-slice-durations-2-5-10-20-days}

\begin{Shaded}
\begin{Highlighting}[]
\NormalTok{traits <-}\StringTok{ }\NormalTok{both %>%}\StringTok{ }
\StringTok{    }\KeywordTok{select}\NormalTok{(ftrait, gtrait) %>%}\StringTok{ }
\StringTok{    }\KeywordTok{as.matrix}\NormalTok{()}
\NormalTok{results2 <-}\StringTok{ }\KeywordTok{roll_cov}\NormalTok{(traits, }\DataTypeTok{width =} \DecValTok{2}\NormalTok{, }\DataTypeTok{center =} \OtherTok{FALSE}\NormalTok{)}
\NormalTok{covariances2 <-}\StringTok{ }\KeywordTok{as.data.frame}\NormalTok{(results2) %>%}\StringTok{ }
\StringTok{    }\KeywordTok{select}\NormalTok{(}\KeywordTok{contains}\NormalTok{(}\StringTok{"f"}\NormalTok{)) %>%}\StringTok{ }
\StringTok{    }\KeywordTok{rownames_to_column}\NormalTok{(}\DataTypeTok{var =} \StringTok{"covariance"}\NormalTok{) %>%}\StringTok{ }
\StringTok{    }\KeywordTok{filter}\NormalTok{(covariance ==}\StringTok{ "gtrait"}\NormalTok{) %>%}\StringTok{ }
\StringTok{    }\KeywordTok{gather}\NormalTok{(}\DataTypeTok{key =} \NormalTok{time_point, }\DataTypeTok{value =} \NormalTok{covariance) %>%}\StringTok{ }
\StringTok{    }\KeywordTok{mutate}\NormalTok{(}\DataTypeTok{time_point =} \KeywordTok{str_replace_all}\NormalTok{(time_point, }\StringTok{"[a-z, .]"}\NormalTok{, }\StringTok{""}\NormalTok{)) %>%}\StringTok{ }
\StringTok{    }\KeywordTok{mutate}\NormalTok{(}\DataTypeTok{time_point =} \KeywordTok{as.numeric}\NormalTok{(time_point)) %>%}\StringTok{ }
\StringTok{    }\KeywordTok{rename}\NormalTok{(}\DataTypeTok{covariance2 =} \NormalTok{covariance)}

\NormalTok{results5 <-}\StringTok{ }\KeywordTok{roll_cov}\NormalTok{(traits, }\DataTypeTok{width =} \DecValTok{5}\NormalTok{, }\DataTypeTok{center =} \OtherTok{FALSE}\NormalTok{)}
\NormalTok{covariances5 <-}\StringTok{ }\KeywordTok{as.data.frame}\NormalTok{(results5) %>%}\StringTok{ }
\StringTok{    }\KeywordTok{select}\NormalTok{(}\KeywordTok{contains}\NormalTok{(}\StringTok{"f"}\NormalTok{)) %>%}\StringTok{ }
\StringTok{    }\KeywordTok{rownames_to_column}\NormalTok{(}\DataTypeTok{var =} \StringTok{"covariance"}\NormalTok{) %>%}\StringTok{ }
\StringTok{    }\KeywordTok{filter}\NormalTok{(covariance ==}\StringTok{ "gtrait"}\NormalTok{) %>%}\StringTok{ }
\StringTok{    }\KeywordTok{gather}\NormalTok{(}\DataTypeTok{key =} \NormalTok{time_point, }\DataTypeTok{value =} \NormalTok{covariance) %>%}\StringTok{ }
\StringTok{    }\KeywordTok{mutate}\NormalTok{(}\DataTypeTok{time_point =} \KeywordTok{str_replace_all}\NormalTok{(time_point, }\StringTok{"[a-z, .]"}\NormalTok{, }\StringTok{""}\NormalTok{)) %>%}\StringTok{ }
\StringTok{    }\KeywordTok{mutate}\NormalTok{(}\DataTypeTok{time_point =} \KeywordTok{as.numeric}\NormalTok{(time_point)) %>%}\StringTok{ }
\StringTok{    }\KeywordTok{rename}\NormalTok{(}\DataTypeTok{covariance5 =} \NormalTok{covariance)}

\NormalTok{results10 <-}\StringTok{ }\KeywordTok{roll_cov}\NormalTok{(traits, }\DataTypeTok{width =} \DecValTok{10}\NormalTok{, }\DataTypeTok{center =} \OtherTok{FALSE}\NormalTok{)}
\NormalTok{covariances10 <-}\StringTok{ }\KeywordTok{as.data.frame}\NormalTok{(results10) %>%}\StringTok{ }
\StringTok{    }\KeywordTok{select}\NormalTok{(}\KeywordTok{contains}\NormalTok{(}\StringTok{"f"}\NormalTok{)) %>%}\StringTok{ }
\StringTok{    }\KeywordTok{rownames_to_column}\NormalTok{(}\DataTypeTok{var =} \StringTok{"covariance"}\NormalTok{) %>%}\StringTok{ }
\StringTok{    }\KeywordTok{filter}\NormalTok{(covariance ==}\StringTok{ "gtrait"}\NormalTok{) %>%}\StringTok{ }
\StringTok{    }\KeywordTok{gather}\NormalTok{(}\DataTypeTok{key =} \NormalTok{time_point, }\DataTypeTok{value =} \NormalTok{covariance) %>%}\StringTok{ }
\StringTok{    }\KeywordTok{mutate}\NormalTok{(}\DataTypeTok{time_point =} \KeywordTok{str_replace_all}\NormalTok{(time_point, }\StringTok{"[a-z, .]"}\NormalTok{, }\StringTok{""}\NormalTok{)) %>%}\StringTok{ }
\StringTok{    }\KeywordTok{mutate}\NormalTok{(}\DataTypeTok{time_point =} \KeywordTok{as.numeric}\NormalTok{(time_point)) %>%}\StringTok{ }
\StringTok{    }\KeywordTok{rename}\NormalTok{(}\DataTypeTok{covariance10 =} \NormalTok{covariance)}

\NormalTok{results20 <-}\StringTok{ }\KeywordTok{roll_cov}\NormalTok{(traits, }\DataTypeTok{width =} \DecValTok{20}\NormalTok{, }\DataTypeTok{center =} \OtherTok{FALSE}\NormalTok{)}
\NormalTok{covariances20 <-}\StringTok{ }\KeywordTok{as.data.frame}\NormalTok{(results20) %>%}\StringTok{ }
\StringTok{    }\KeywordTok{select}\NormalTok{(}\KeywordTok{contains}\NormalTok{(}\StringTok{"f"}\NormalTok{)) %>%}\StringTok{ }
\StringTok{    }\KeywordTok{rownames_to_column}\NormalTok{(}\DataTypeTok{var =} \StringTok{"covariance"}\NormalTok{) %>%}\StringTok{ }
\StringTok{    }\KeywordTok{filter}\NormalTok{(covariance ==}\StringTok{ "gtrait"}\NormalTok{) %>%}\StringTok{ }
\StringTok{    }\KeywordTok{gather}\NormalTok{(}\DataTypeTok{key =} \NormalTok{time_point, }\DataTypeTok{value =} \NormalTok{covariance) %>%}\StringTok{ }
\StringTok{    }\KeywordTok{mutate}\NormalTok{(}\DataTypeTok{time_point =} \KeywordTok{str_replace_all}\NormalTok{(time_point, }\StringTok{"[a-z, .]"}\NormalTok{, }\StringTok{""}\NormalTok{)) %>%}\StringTok{ }
\StringTok{    }\KeywordTok{mutate}\NormalTok{(}\DataTypeTok{time_point =} \KeywordTok{as.numeric}\NormalTok{(time_point)) %>%}\StringTok{ }
\StringTok{    }\KeywordTok{rename}\NormalTok{(}\DataTypeTok{covariance20 =} \NormalTok{covariance)}



\NormalTok{all_cov <-}\StringTok{ }\KeywordTok{left_join}\NormalTok{(covariances2, covariances5) %>%}\StringTok{ }
\StringTok{    }\KeywordTok{left_join}\NormalTok{(covariances10) %>%}\StringTok{ }
\StringTok{    }\KeywordTok{left_join}\NormalTok{(covariances20) %>%}\StringTok{ }
\StringTok{    }\KeywordTok{gather}\NormalTok{(}\DataTypeTok{key =} \NormalTok{slice, }\DataTypeTok{value =} \NormalTok{covariance, }\DecValTok{2}\NormalTok{:}\DecValTok{5}\NormalTok{) %>%}\StringTok{ }
\StringTok{    }\KeywordTok{mutate}\NormalTok{(}\DataTypeTok{slice =} \KeywordTok{str_replace_all}\NormalTok{(slice, }\StringTok{"[a-z, .]"}\NormalTok{, }\StringTok{""}\NormalTok{)) %>%}\StringTok{ }
\StringTok{    }\KeywordTok{rename}\NormalTok{(}\DataTypeTok{time_slice =} \NormalTok{slice) %>%}\StringTok{ }
\StringTok{    }\KeywordTok{mutate}\NormalTok{(}\DataTypeTok{time_slice =} \KeywordTok{as.numeric}\NormalTok{(time_slice))}


\NormalTok{cov_plot <-}\StringTok{ }\NormalTok{all_cov %>%}\StringTok{ }
\StringTok{    }\KeywordTok{ggplot}\NormalTok{(}\KeywordTok{aes}\NormalTok{(}\DataTypeTok{x =} \NormalTok{time_point, }\DataTypeTok{y =} \NormalTok{covariance, }\DataTypeTok{color =} \KeywordTok{factor}\NormalTok{(time_slice), }\DataTypeTok{group =} \NormalTok{time_slice)) +}\StringTok{ }\KeywordTok{geom_line}\NormalTok{(}\DataTypeTok{size =} \DecValTok{1}\NormalTok{) +}
\StringTok{     }\KeywordTok{scale_color_viridis_d}\NormalTok{(}\DataTypeTok{name =} \StringTok{"Duration of time slice"}\NormalTok{) +}\StringTok{ }\KeywordTok{geom_hline}\NormalTok{(}\DataTypeTok{yintercept =} \DecValTok{0}\NormalTok{) +}\StringTok{ }\KeywordTok{ylab}\NormalTok{(}\StringTok{"Cov[f(T), g(T)]"}\NormalTok{) +}
\StringTok{    }\KeywordTok{xlab}\NormalTok{(}\StringTok{"Time (days)"}\NormalTok{) +}\StringTok{ }\KeywordTok{theme}\NormalTok{(}\DataTypeTok{legend.position =} \StringTok{"top"}\NormalTok{)}
\end{Highlighting}
\end{Shaded}

\paragraph{Plot all together}\label{plot-all-together}

\begin{Shaded}
\begin{Highlighting}[]
\NormalTok{tpcs}
\end{Highlighting}
\end{Shaded}

\includegraphics{08_covariance_messing_around_files/figure-latex/unnamed-chunk-9-1.pdf}

\begin{Shaded}
\begin{Highlighting}[]
\NormalTok{all_plots1 <-}\StringTok{ }\KeywordTok{plot_grid}\NormalTok{(temp_plot, trait_plot, cov_plot, }\DataTypeTok{nrow =} \DecValTok{3}\NormalTok{, }\DataTypeTok{ncol =} \DecValTok{1}\NormalTok{)}
\NormalTok{all_plots1}
\end{Highlighting}
\end{Shaded}

\includegraphics{08_covariance_messing_around_files/figure-latex/unnamed-chunk-10-1.pdf}

\subsubsection{\texorpdfstring{Now, focusing on the 2 day time slice
only, check that \texttt{\textless{}R0(T)\textgreater{}} =
\texttt{\textless{}f(T)g(T)\textgreater{}} =
\texttt{\textless{}f(T)\textgreater{}\textless{}g(T)\textgreater{}} +
\texttt{Cov{[}f(T),\ g(T){]}}}{Now, focusing on the 2 day time slice only, check that \textless{}R0(T)\textgreater{} = \textless{}f(T)g(T)\textgreater{} = \textless{}f(T)\textgreater{}\textless{}g(T)\textgreater{} + Cov{[}f(T), g(T){]}}}\label{now-focusing-on-the-2-day-time-slice-only-check-that-r0t-ftgt-ftgt-covft-gt}

Looks like this isn't working as I expected! i.e.~the blue and green
lines below should be lining up\ldots{}Hmmm\ldots{}.

\begin{Shaded}
\begin{Highlighting}[]
\NormalTok{### get time rolling average for R_0, i.e. <R0(T)> =  <f(T)g(T)>}
\NormalTok{R_0 =}\StringTok{ }\NormalTok{both %>%}\StringTok{ }
\StringTok{    }\KeywordTok{mutate}\NormalTok{(}\DataTypeTok{R_0 =} \NormalTok{ftrait*gtrait) %>%}\StringTok{ }
\StringTok{    }\KeywordTok{select}\NormalTok{(R_0) %>%}\StringTok{ }
\StringTok{    }\KeywordTok{as.matrix}\NormalTok{()}
\NormalTok{results2_R_0 <-}\StringTok{ }\KeywordTok{roll_mean}\NormalTok{(R_0, }\DataTypeTok{width =} \DecValTok{2}\NormalTok{)}
\NormalTok{mean2 <-}\StringTok{ }\KeywordTok{data.frame}\NormalTok{(}\DataTypeTok{mean_R0 =} \NormalTok{results2_R_0, }\DataTypeTok{complete_obs =} \OtherTok{TRUE}\NormalTok{) %>%}\StringTok{ }
\StringTok{    }\KeywordTok{rownames_to_column}\NormalTok{(}\DataTypeTok{var =} \StringTok{"time"}\NormalTok{)}


\NormalTok{### get time rolling average for f(T), <f(T)>}
\NormalTok{f_avg <-}\StringTok{ }\NormalTok{both %>%}\StringTok{ }
\StringTok{    }\KeywordTok{select}\NormalTok{(ftrait) %>%}\StringTok{ }
\StringTok{    }\KeywordTok{as.matrix}\NormalTok{()}
\NormalTok{results2_fmean <-}\StringTok{ }\KeywordTok{roll_mean}\NormalTok{(f_avg, }\DataTypeTok{width =} \DecValTok{2}\NormalTok{)}
\NormalTok{meanf2 <-}\StringTok{ }\KeywordTok{data.frame}\NormalTok{(}\DataTypeTok{mean_f =} \NormalTok{results2_fmean, }\DataTypeTok{complete_obs =} \OtherTok{TRUE}\NormalTok{) %>%}\StringTok{ }
\StringTok{    }\KeywordTok{rownames_to_column}\NormalTok{(}\DataTypeTok{var =} \StringTok{"time"}\NormalTok{)}

\NormalTok{### get time rolling average for g(T), <g(T)>}
\NormalTok{g_avg <-}\StringTok{ }\NormalTok{both %>%}\StringTok{ }
\StringTok{    }\KeywordTok{select}\NormalTok{(gtrait) %>%}\StringTok{ }
\StringTok{    }\KeywordTok{as.matrix}\NormalTok{()}
\NormalTok{results2_gmean <-}\StringTok{ }\KeywordTok{roll_mean}\NormalTok{(g_avg, }\DataTypeTok{width =} \DecValTok{2}\NormalTok{, }\DataTypeTok{complete_obs =} \OtherTok{TRUE}\NormalTok{)}
\NormalTok{meang2 <-}\StringTok{ }\KeywordTok{data.frame}\NormalTok{(}\DataTypeTok{mean_g =} \NormalTok{results2_gmean) %>%}\StringTok{ }
\StringTok{    }\KeywordTok{rownames_to_column}\NormalTok{(}\DataTypeTok{var =} \StringTok{"time"}\NormalTok{)}


\NormalTok{### put all the averages together}
\NormalTok{all_avgs <-}\StringTok{ }\KeywordTok{left_join}\NormalTok{(mean2, meanf2) %>%}\StringTok{ }
\StringTok{    }\KeywordTok{left_join}\NormalTok{(meang2) %>%}\StringTok{ }
\StringTok{    }\KeywordTok{mutate}\NormalTok{(}\DataTypeTok{time =} \KeywordTok{as.numeric}\NormalTok{(time)) %>%}\StringTok{ }
\StringTok{    }\KeywordTok{left_join}\NormalTok{(covariances2, }\DataTypeTok{by =} \KeywordTok{c}\NormalTok{(}\StringTok{"time"} \NormalTok{=}\StringTok{ "time_point"}\NormalTok{)) %>%}\StringTok{ }
\StringTok{    }\KeywordTok{mutate}\NormalTok{(}\StringTok{`}\DataTypeTok{<f(T)><g(T)> + Cov[f(T), g(T)]}\StringTok{`} \NormalTok{=}\StringTok{ }\NormalTok{(ftrait*gtrait) +}\StringTok{ }\NormalTok{covariance2) %>%}\StringTok{ }
\StringTok{    }\KeywordTok{select}\NormalTok{(-complete_obs) %>%}\StringTok{ }
\StringTok{    }\KeywordTok{mutate}\NormalTok{(}\StringTok{`}\DataTypeTok{<f(T)><g(T)>}\StringTok{`} \NormalTok{=}\StringTok{ }\NormalTok{ftrait*gtrait) %>%}\StringTok{ }
\StringTok{    }\KeywordTok{gather}\NormalTok{(}\DataTypeTok{key =} \NormalTok{parameter, }\DataTypeTok{value =} \NormalTok{value, }\DecValTok{2}\NormalTok{:}\DecValTok{7}\NormalTok{)}

\NormalTok{## plot}
\NormalTok{all_avgs %>%}\StringTok{ }
\StringTok{    }\KeywordTok{filter}\NormalTok{(parameter %in%}\StringTok{ }\KeywordTok{c}\NormalTok{(}\StringTok{"<f(T)><g(T)> + Cov[f(T), g(T)]"}\NormalTok{, }\StringTok{"R_0"}\NormalTok{, }\StringTok{"<f(T)><g(T)>"}\NormalTok{, }\StringTok{"covariance2"}\NormalTok{)) %>%}\StringTok{ }
\StringTok{    }\KeywordTok{mutate}\NormalTok{(}\DataTypeTok{parameter =} \KeywordTok{str_replace}\NormalTok{(parameter, }\StringTok{"R_0"}\NormalTok{, }\StringTok{"<f(T)g(T)>"}\NormalTok{)) %>%}\StringTok{ }
\StringTok{    }\KeywordTok{mutate}\NormalTok{(}\DataTypeTok{parameter =} \KeywordTok{str_replace}\NormalTok{(parameter, }\StringTok{"covariance2"}\NormalTok{, }\StringTok{"Cov[f(T), g(T)]"}\NormalTok{)) %>%}\StringTok{ }
\StringTok{    }\KeywordTok{ggplot}\NormalTok{(}\KeywordTok{aes}\NormalTok{(}\DataTypeTok{x =} \NormalTok{time, }\DataTypeTok{y =} \NormalTok{value, }\DataTypeTok{color =} \NormalTok{parameter)) +}\StringTok{ }\KeywordTok{geom_line}\NormalTok{() }
\end{Highlighting}
\end{Shaded}

\includegraphics{08_covariance_messing_around_files/figure-latex/unnamed-chunk-11-1.pdf}


\end{document}
